%Dokumentenklasse, andere Beispiele: Statt llncs "`article"' "`report"', "`book"'...
%Achtung llncs benötigt die Datei llncs.cls im gleichen Ordner
\documentclass[runningheads]{article}

%sorgt für deutsche Übersetzungen, z.B. "`Literatur"' statt "`References"'
\usepackage[ngerman]{babel}
\usepackage[utf8]{inputenc}
%Ermöglicht das Einbinden von Bildern
\usepackage{graphicx}
%Darstellung von URLs
\usepackage{url}
\usepackage{listings}

\usepackage{amsmath}
\usepackage{amsfonts}
\usepackage{amssymb}
\usepackage{amstext}


\begin{document}

%Benötigte Angaben für die Titelseite
\title{Manual zu Aufgabe 2 des praktischen Übungsblatts}
%Hier Zeilenumbruch durch \\, da \newline in dieser Umgebung nicht funktioniert.
\author{Christopher Schmidt, Maren Pielka}


%Erstellung des Titels
\maketitle

\section{Anforderungen}
\begin{itemize}
\item Python 2.6
\item Python-Pakete:
\begin{itemize}
\item Statistics
\item Matplotlib
\item Argparse
\end{itemize}
Die Python-Pakete lassen sich mit folgendem Befehl installieren:
\begin{lstlisting} 
pip install statistics matplotlib argparse
\end{lstlisting}
\end{itemize}

\section{Bedienung}
\begin{itemize}
\item Das Programm wird über die Konsole mit folgender Eingabe aufgerufen: \\
\begin{lstlisting}
./execute_exercise2.sh
\end{lstlisting}
\item Parameter, um den Zeitraum einzugrenzen:
\begin{lstlisting}
./execute_exercise2.sh --low YYYY-MM-DD --high YYYY-MM-DD
\end{lstlisting}
z.B:
\begin{lstlisting}
./execute_exercise2.sh --low 2011-01-01 --high 2014-12-31
\end{lstlisting}
\item Beispielausgabe:
\begin{lstlisting}
Minimum:  3.544
Maximum:  82.85
Mean:  15.5133651461
Deviation:  5.52623789293
Sections: 1378899362 1392371162 1411025762 1421048162 1445589362
\end{lstlisting}
\end{itemize}



\end{document}
