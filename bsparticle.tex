%Dokumentenklasse, andere Beispiele: Statt llncs "`article"' "`report"', "`book"'...
%Achtung llncs ben�tigt die Datei llncs.cls im gleichen Ordner
\documentclass[runningheads]{llncs}

%sorgt f�r deutsche �bersetzungen, z.B. "`Literatur"' statt "`References"'
\usepackage{ngerman}
%Erm�glicht das Einbinden von Bildern
\usepackage{graphicx}
%Darstellung deutscher Sonderzeichen (Umlaute etc)
\usepackage[ansinew]{inputenc}
%Darstellung von URLs
\usepackage{url}

\usepackage{amsmath}
\usepackage{amsfonts}
\usepackage{amssymb}
\usepackage{amstext}

%Spezielle Einbindung einer URL zur Angabe in der caption
\urldef \urlSchnab \url{http://upload.wikimedia.org/wikipedia/commons/f/f2/Platypus.jpg}

\begin{document}

%Ben�tigte Angaben f�r die Titelseite
\title{Insert\_Title\_Here}
%Hier Zeilenumbruch durch \\, da \newline in dieser Umgebung nicht funktioniert.
\author{Insert\_Name\_Here \\ Matrikel-Nr: NaN}
%Hier das Institut angeben
\institute{Schnabeltier Universit�t}

%Erstellung des Titels
\maketitle

\begin{abstract}
Sehr kurze Zusammenfassung (1-3 S�tze) des Textes in welchen Aufgabe, vorgehensweise und Ergebnis vorgestellt werden.
\end{abstract}

\section{Einleitung}
Hier wird eine Einleitung platziert

%Erster Abschnitt, in geschweiften Klammern Titel des ersten Abschnittes
\section{Erster Abschnitt}

Hier wird der eigene Text eingef�gt, der in den ersten Abschnitt passt.

%Erster Unterabschnitt des Ersten Abschnittes
\subsection{Erster Unterabschnitt}

Der Inhalt diesen Satzes stammt aus Quelle 1, daher m�ssen wir auf sie referenzieren. [1] \newline
Ein Unterabschnitt ist definiert als kleinerer Abschnitt in einem gr��eren. [2]

%Einbinden eines Bildes
%In eckigen Klammern: Platzierung, h = "`here"', b = "`bottom"', t = "`top"'
\begin{figure}[ht]
	%zentrieren
	\centering
	%Abgabe des Pfads des Bildes, in eckigen Klammern die gew�nschte Gr��e
	\includegraphics[width=7cm]{Schnabeltier.jpg}
	%Bildunterschrift
	\caption{Schnabeltier, Quelle: \urlSchnab{}}
	%Zur Referenzierung auf das Bild ben�tigt
	\label{fig:Schnab}
\end{figure}

%Verhindert Einr�ckung des Absatzes
\noindent
Der wissenschaftliche Name des Schnabeltiers lautet "`Ornithorhynchus anatinus"'. W�hrend das Schnabeltier-M�nnchen eine Kopf-Rumpf-L�nge von 40-60 cm aufweist, ist das Weibchen mit 39-55 cm kleiner. [3] \newline
Der Name "`Schnabeltier"' geht auf den Schnabel zur�ck, der �hnlichkeiten mit dem einer Ente aufweist. Dabei haben erwachsene Schnabeltiere statt Z�hnen Hornplatten an Ober- und Unterkiefern, mit denen sie in der Lage sind, Nahrung zu zermahlen. Die Jungen besitzen bei der Geburt noch Backenz�hne, diese verlieren sie allerdings w�hrend ihrer Entwicklung. \cite{4} \newline
Das ungew�hnliche Aussehen des Schnabeltiers spiegelt sich auch in folgender Aussage wieder: "`A disbeliever in anything beyond his own reason, might exclaim: Surely two distinct creators must have been at work."' Charles Darwin. [4]

%Verhindert Einr�ckung des Absatzes
\noindent
%�ber \ref wird auf das mit \label markierte Objekt referenziert (kann auch section, table etc. sein)
% ! Achtung, bei Verwendung von \ref o.�. kann mehrmaliges kompilieren n�tig werden!
In Abbildung \ref{fig:Schnab} sieht man ein solches Schnabeltier.

\section{Zweiter Abschnitt}

Hier geht es um die Normalverteilung, die durch Formel \ref{eq} beschrieben wird.

% Umgebung f�r eine Gleichung
\begin{equation*}
	\mathcal{N}(x | \mu, \sigma) = \frac{1}{\sigma \sqrt{2 \pi}} \cdot \mbox{exp}\left( - \frac{1}{2} \left(\frac{x - \mu }{\sigma}\right) \right)
	\label{eq}
\end{equation*} 

\noindent
% $ beginnt eine mathematische Umgebung im Text. Unbedingt wieder mit $ schlie�en!
Darin bezeichnet $x$ die Zufallsvariable, $\mu$ den Erwartungswert und $\sigma$ die Standardabweichung. Au�erdem k�nnen wir auch eine mathematische Umgebung im Text benutzen: $\sum_{i=0}^{42} i = 903$.

\section{Schlusswort}
hier wird ein Schlusswort oder eine Zusammenfassung eingef�gt

%Literaturverzeichnis mit 4 Eintr�gen
\begin{thebibliography}{4}

%Erster Literatureintrag
\bibitem[1]{1}
Autor Autoris, La TexUser, "`Titel der Quelle aus der zitiert werden soll"' in \textit{Tagungsband der gro�en internationalen Konferenz}, 2020.

%Zweiter Literatureintrag
\bibitem[2]{2}
Duden, \url{http://www.duden.de/rechtschreibung/Unterabschnitt}, abgerufen am 7.10.2012.

%Dritter Literatureintrag
\bibitem[3]{3}
Tierlexikon, \url{http://www.das-tierlexikon.de/schnabeltier-245-pictures.htm}, abgerufen am 7.10.2012.

%Vierter Literatureintrag
\bibitem[4]{4}
Wikipedia, \url{http://de.wikipedia.org/wiki/Schnabeltier}, abgerufen am 7.10.2012.

\end{thebibliography}

\end{document}
